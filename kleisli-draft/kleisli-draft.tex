\documentclass[12pt]{article}

%{{{ Preamble

%%%%%%%%{{{ Packages
%%%%%%%%%%%%%%%%%%%%

% the default margins have always felt big to me
\usepackage[margin=1in]{geometry}

% we aren't animals - we might use more than ASCII
\usepackage[utf8]{inputenc}
\usepackage[T1]{fontenc}

% we're gay, we like colors
\usepackage[dvipsnames]{xcolor}

% obligatory math environments, symbols, and theorems
\usepackage{amsmath, amssymb, amsthm}

% moar symbols
\usepackage{stmaryrd}

% obligatory citation library
\usepackage{natbib}

% sometimes you gotta draw stuff, like _c_ommutative _d_iagrams.
\usepackage{tikz, tikz-cd}

% sometimes you gotta put in pretty pictures
\usepackage{graphicx}

% sometimes you gotta write code
\usepackage{listings}

% proof trees are useful
\usepackage{proof}

% use \mathbbm for bb-numerals, use \bm for bold math symbols.
\usepackage{bbm, bm}

% I like clickable links in pdfs
\usepackage{hyperref}

% Convenient todo-notes and missing figure boxes
\usepackage{todonotes}

% remove paragraph indentation
\usepackage{parskip}

% allow fancy stuff (custom headers and footers)
\usepackage{fancyhdr}

% pretty boxes are pretty
\usepackage[framemethod=TikZ]{mdframed}

%}}}

%%%%%%%%{{{ Formatting
%%%%%%%%%%%%%%%%%%%%%%

% prevent orphans/widows
\clubpenalty = 10000
\widowpenalty = 10000

% never break words across lines -- hyphens are stupid
\hyphenpenalty 10000
\exhyphenpenalty 10000

%}}}

%%%%%%%%{{{ Environments
%%%%%%%%%%%%%%%%%%%%%%%%

\newtheorem{thm}{Theorem}

\theoremstyle{definition}
\newtheorem{defn}{Definition}

%}}}

%%{{{ Aliases and Commands
%%%%%%%%%%%%%%%%%%%%%%%%%%

%{{{ blackboard letters
\newcommand{\N}{\mathbb{N}}
\newcommand{\Z}{\mathbb{Z}}
\newcommand{\Q}{\mathbb{Q}}
\newcommand{\R}{\mathbb{R}}
\newcommand{\C}{\mathbb{C}}
%}}}

%{{{ categories
\newcommand*{\catFont}[1]{\mathsf{#1}} 
\newcommand*{\catVarFont}[1]{\mathcal{#1}}

\newcommand{\Set}{\catFont{Set}}
\newcommand{\Top}{\catFont{Top}}
\newcommand{\Cat}{\catFont{Cat}}
\newcommand{\Grp}{\catFont{Grp}}
\newcommand{\Gph}{\catFont{Gph}}
\newcommand{\Mod}{\catFont{Mod}}
\newcommand{\Sub}{\catFont{Sub}}
\newcommand{\FP}{\catFont{FP}}
\newcommand{\Pos}{\catFont{Pos}}
\newcommand{\FinSet}{\catFont{FinSet}}

\newcommand{\cata}{\catVarFont{A}}
\newcommand{\catb}{\catVarFont{B}}
\newcommand{\catc}{\catVarFont{C}}
\newcommand{\catd}{\catVarFont{D}}
\newcommand{\catx}{\catVarFont{X}}
\newcommand{\caty}{\catVarFont{Y}}
\newcommand{\catZ}{\catVarFont{Z}}
%}}}

%{{{ "operators" (words in math mode)
\DeclareMathOperator{\Hom}{Hom}
\DeclareMathOperator{\End}{End}
\DeclareMathOperator{\Aut}{Aut}
\DeclareMathOperator{\im}{im}
\DeclareMathOperator{\coker}{coker}
%}}}

%{{{ arrows
\newcommand{\hookr}{\hookrightarrow}
\newcommand{\hookl}{\hookleftarrow}
\newcommand{\monor}{\rightarrowtail}
\newcommand{\monol}{\leftarrowtail}
\newcommand{\epir}{\twoheadrightarrow}
\newcommand{\epil}{\twoheadleftarrow}
\newcommand{\regepir}{\rightarrowtriangle}
\newcommand{\regepil}{\leftarrowtriangle}
%}}}

%{{{ categorical limits
\newcommand{\limr}{\varinjlim}
\newcommand{\liml}{\varprojlim}
%}}}

%{{{ analysis
\DeclareMathOperator{\dif}{d \!}
\DeclareMathOperator{\Dif}{D \!}
\newcommand{\del}{\partial}
\newcommand*{\abs}[1]{\left \lvert #1 \right \rvert}
\newcommand*{\norm}[1]{\left \lVert #1 \right \rVert}
\newcommand*{\eval}[1]{\left . #1 \right \rvert}

% partial derivative command (taken from commath package)
% usage: \pd[n]{f}{x}
\newcommand*{\pd}[3][]{\ensuremath{
\ifinner
\tfrac{\partial{^{#1}}#2}{ \partial{#3^{#1}} }
\else
\dfrac{\partial{^{#1}}#2}{ \partial{#3^{#1}} }
\fi
}}

% ordinary derivative command (taken from commath package)
% usage: \od[n]{f}{x}
\newcommand*{\od}[3][]{\ensuremath{
\ifinner
\tfrac{\dif{^{#1}}#2}{ \dif{#3^{#1}} }
\else
\dfrac{\dif{^{#1}}#2}{ \dif{#3^{#1}} }
\fi
}}

%}}}

%{{{ algebra
\newcommand{\meet}{\wedge}
\newcommand{\join}{\vee}
\newcommand{\id}{\mathrm{id}}
%}}}

%{{{ topology
\newcommand*{\interior}[1]{ {\kern0pt#1}^{\mathrm{o}} }
%}}}

%{{{ logic
\renewcommand{\diamond}{\lozenge}
\newcommand*{\denote}[1]{\llbracket #1 \rrbracket}

% \godelnum command, stolen from 
% https://www.logicmatters.net/
% 	latex-for-logicians/symbols/corner-quotes-for-godel-numbers/
\newbox\gnBoxA
\newdimen\gnCornerHgt
\setbox\gnBoxA=\hbox{$\ulcorner$}
\global\gnCornerHgt=\ht\gnBoxA
\newdimen\gnArgHgt
\def\godelnum #1{%
\setbox\gnBoxA=\hbox{$#1$}%
\gnArgHgt=\ht\gnBoxA%
\ifnum     \gnArgHgt<\gnCornerHgt \gnArgHgt=0pt%
\else \advance \gnArgHgt by -\gnCornerHgt%
\fi \raise\gnArgHgt\hbox{$\ulcorner$} \box\gnBoxA %
\raise\gnArgHgt\hbox{$\urcorner$}}

% coding function
\newcommand*{\code}[1]{\langle #1 \rangle}

%}}}

%{{{ misc symbols
\newcommand{\teq}{\triangleq}
\newcommand{\fin}{ \subseteq_{\text{fin}} }

% important words
\newcommand*{\important}[1]{\textcolor{MidnightBlue}{#1}}

% define a "danger" symbol for use when something surprising might occur
% https://tex.stackexchange.com/questions/159669/
% 	how-to-print-a-warning-sign-triangle-with-exclamation-point
% use outside of math mode!

\newcommand*{\TakeFourierOrnament}[1]{}}

%}}}

%% project specific aliases (if they exist)
\IfFileExists{../preamble.tex}{\input{../preamble.tex}}{}

%}}}

%% Heading
\author{Chris Grossack}
\title{Kleisli Categories and You: Applications to Raags}

\begin{document}
\maketitle

\section{Intro}
  

Recall the setup. We have a pair of adjoint functors
$A \dashv C : \Grp \to \Gph$
which induce a monad $CA : \Gph \to \Gph$.
We will be working off of analogy with the free forgetful
adjunction $F \dashv U : \Grp \to \Set$. 

In the same way $U$ lets us view groups in terms of their underlying set,
$C$ lets us view groups in terms of their ``underlying graph''. Similarly,
since $F$ lets us construct a free group on a set, $A$ lets us construct
a ``free group on a graph'' -- a raag.

Obviously there's some superficial similarities. If we identify
a set with its discrete graph, for instance, then the two notions of
free group coincide. But how precise can we actually make this analogy?
The answer can be found in Chapter VI of "Categories for the Working Mathematician".
To discuss it, though, we need to talk about \important{Algebras}. 
These are generalizations of the following observation:

\section{Algebras of $F \dashv U$}

Let $G$ be the underlying set of a group. The free group $FG$ gives us
all of the ways to write down ``formal products'' of elements of $G$. 
But since $G$ admits a group structure, $G$ comes equipped with a way to
evalute these formal products to get an actual element of $G$. That is,
$G$ comes equipped with a map $\alpha : UFG \to G$, which we can view as
this evaluation map. There are two conditions on $\alpha$ which are implied
by the group structure on $G$:

First, $\alpha$ should be a retract of the inclusion $\eta : G \hookrightarrow UFG$ which sends $g$ to the word $[g]$. 

Second, if we have a word of words (an element of $UFUFG$), then there are two 
natural ways to evaluate this to an element of $G$. They should be the same.
This is probably best given as an example. Let's work with $G = \mathbb{Z}$.

Say $[[1,3],[-2],[1,-3]]$ is a word of words. The first way to get an
element of $G$ is to concatenate these words to get $[1,3,-2,1,-3]$,
then evaluate this with $\alpha$ to get $0$. The second way is to evaluate
each inner word individually, to get $[4,-2,-2]$, then evaluate the outer word
to get $0$. 

We can describe the first condition with the commutative diagram

\begin{center}
\begin{tikzcd}
G \arrow[r, "\eta"] \arrow[rd, "\text{Id}"'] & UFG \arrow[d, "\alpha"] \\
                                             & G                      
\end{tikzcd}
\end{center}

And we can describe the second as

\begin{center}
  \begin{tikzcd}
    UFUFG \arrow[r, "UF\alpha"] \arrow[d, "\mu"] & UFG \arrow[d, "\alpha"]\\
    UFG \arrow[r,"\alpha"]                       & G
  \end{tikzcd}
\end{center}

In fact, it's not hard to convince yourself that the algebras of the monad
$UF : \Set \to \Set$ are exactly the groups! Similarly, if you look at the 
free-forgetful adjunction for rings, its algebras are exactly the rings.
Or if you look at the monad $G \times - : \Set \to \Set$ for a fixed group $G$,
its algebras are exactly the $G$-sets.

In all of these cases, the ``algebras'' are exactly the algebras in the 
classical sense which one might expect to be associated to each monad.

\section{Algebras for a General Monad}

So now we generalize, and given any monad $(M : \catc \to \catc, \eta, \mu)$, we say an
\important{$M$-Algebra} is an object $X \in \catc$ equipped with an arrow
$\alpha : MX \to X$ satisfying the above conditions.

Of course there is a category of $M$-algebras, called $\catc^M$, where we define a 
$M$-homomorphism from $(X,\alpha)$ to $(Y,\beta)$ by 
a map on the underlying objects which ``respects the $M$-structure''
in the sense that the following diagram commutes:

\begin{center}
  \begin{tikzcd}
    MX \ar[r, "Mf"] \ar[d, "\alpha"] & MY \ar[d,"\beta"]\\
    X \ar[r,"f"]                     & Y
  \end{tikzcd}
\end{center}

There is an obvious forgetful functor $(X,\alpha) \mapsto X$ from 
$\catc^M \to \catc$. Perhaps less obviously, there is a left adjoint 
to this forgetful functor which sends $X$ to $(MX,\mu)$.

Now for an important definition: 

\begin{defn}
We say an adjunction $L \dashv R : \cata \to \catc$ is \important{Monadic}
whenever $\cata$ and $\catc^{RL}$ are equivalent. Here we want to
think about $\cata$ as the \emph{Category of Algebras} and $\catc$ as the 
\emph{base category}. This is the case in all of the above examples -- 
for instance $\Grp \simeq \Set^{UF}$. In this case, $L$ and $R$ agree
with the free and forgetful functors from the last paragraph.
\end{defn}

\section{Why Should We Care?}

As you may have guessed, our adjunction is monadic.\footnote{%
Nota bene: this is fairly routine to check, but it's also
annoying and I was slightly sloppy when verifying it. If this turns into a paper, I'll be more thorough.} 
This means that $\Gph^{CA} \simeq \Grp$!

First and foremost, this means the analogy between raags and free groups 
is basically perfect. $\Gph$ and $\Set$ share almost all of the same 
categorical properties, and we have a monadic adjunction from $\Grp$ in both
cases. So as far as category theory cares, we might as well be working 
with the same object.

The downside of this realization is it kills any dreams of a nice 
graph theoretic characterization of monos $A\Gamma \monor A\Delta$.
After all, there's no nice set theoretic characterization of monos of 
free groups $FX \monor FY$. Thankfully there are two upsides:

The first is that this basically kills \emph{everyone's} dreams. 
It's not just that there's no nice \emph{category theoretic} description
of free group inclusions. There's no nice description of free group inclusions
using \emph{any} tools. The best we can do is say 
``Images of the generators shouldn't have nontrivial relations''. So 
this theorem provides good evidence that there are no good graph theoretic
characterizations for raag embeddings. So even though we failed, we failed for
good reason, and now we know what that reason is.

The second upside is that Category Theory does have a saving grace for this
problem, but it's a little bit awkward, and I haven't had time to play with
it yet. This is called the \important{Kleisli Category}, and is described in
Chapter VI, Section 5 of ``Categories for the Working Mathematician''.

Before we talk about what a Kleisli Category \emph{is}, I'll point you 
to the following theorem:

\begin{thm}
Given a monad $M : \catc \to \catc$, the Kleisli Category $\catc_M$ is
equivalent to the full subcategory of free objects in $\catc^M$.
\end{thm}

Let that sink in. This (for us) says that the category $\catc_{CA}$ is
equivalent to category of all raags and group homomorphisms! So monos
in $\catc_{CA}$ will be in bijection with monos of raags! This is exactly
what we're looking for!
Ok. So what \emph{is} this magical category? And why was I so pessimistic
earlier if it exists?

Given a monad $M : \catc \to \catc$, the Kleisli Category $\catc_M$ 
has the same objects as $\catc$, but has a strange notion of arrow:

We add a Kleisli Arrow $f : X \to_K Y$ for every arrow $f : X \to MY$ in $\catc$.

The identity arrows are given by $\eta : X \to MX$, and we
compose two Kleisli Arrows according to the following rule:
$(g : Y \to_K Z) \circ_K (f : X \to_K Y)$ is given by

\[ X \overset{f}{\to} MY \overset{Mg}{\to} MMZ \overset{\mu}{\to} MZ \]

So if we can find a characterization of monos $m : \Gamma \monor_K \Delta$,
we would have a characterization of monos $\tilde{m} : A \Gamma \monor A \Delta$.
Unfortunately, no amount of googling has turned up such a characterization 
for free groups, so I assume we'll struggle to interpret this. 

Who knows, though. One attainable goal might be to find a new \emph{class} 
of monos, by finding some sufficient condition in the Kleisli Category.
Even if this doesn't get all the possible monos, this would still be a 
nice result.

I haven't had time to play around with this, so I'm not sure how
achievable this will end up being. Either way, now you're caught up 
on my thoughts.


% \newpage
% \bibliographystyle{plain}
% \bibliography{\string~/.bib.bib}

\end{document}
