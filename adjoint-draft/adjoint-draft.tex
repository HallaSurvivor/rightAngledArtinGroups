\documentclass[12pt]{article}

%{{{ Preamble

%%%%%%%%{{{ Packages
%%%%%%%%%%%%%%%%%%%%

% the default margins have always felt big to me
\usepackage[margin=1in]{geometry}

% we aren't animals - we might use more than ASCII
\usepackage[utf8]{inputenc}
\usepackage[T1]{fontenc}

% we're gay, we like colors
\usepackage[dvipsnames]{xcolor}

% obligatory math environments, symbols, and theorems
\usepackage{amsmath, amssymb, amsthm}

% moar symbols
\usepackage{stmaryrd}

% obligatory citation library
\usepackage{natbib}

% sometimes you gotta draw stuff, like _c_ommutative _d_iagrams.
\usepackage{tikz, tikz-cd}

% sometimes you gotta put in pretty pictures
\usepackage{graphicx}

% sometimes you gotta write code
\usepackage{listings}

% proof trees are useful
\usepackage{proof}

% use \mathbbm for bb-numerals, use \bm for bold math symbols.
\usepackage{bbm, bm}

% I like clickable links in pdfs
\usepackage{hyperref}

% Convenient todo-notes and missing figure boxes
\usepackage{todonotes}

% remove paragraph indentation
\usepackage{parskip}

% allow fancy stuff (custom headers and footers)
\usepackage{fancyhdr}

% pretty boxes are pretty
\usepackage[framemethod=TikZ]{mdframed}

%}}}

%%%%%%%%{{{ Formatting
%%%%%%%%%%%%%%%%%%%%%%

% prevent orphans/widows
\clubpenalty = 10000
\widowpenalty = 10000

% never break words across lines -- hyphens are stupid
\hyphenpenalty 10000
\exhyphenpenalty 10000

%}}}

%%%%%%%%{{{ Environments
%%%%%%%%%%%%%%%%%%%%%%%%

\newtheorem{thm}{Theorem}

\theoremstyle{definition}
\newtheorem{defn}{Definition}

%}}}

%%{{{ Aliases and Commands
%%%%%%%%%%%%%%%%%%%%%%%%%%

%{{{ blackboard letters
\newcommand{\N}{\mathbb{N}}
\newcommand{\Z}{\mathbb{Z}}
\newcommand{\Q}{\mathbb{Q}}
\newcommand{\R}{\mathbb{R}}
\newcommand{\C}{\mathbb{C}}
%}}}

%{{{ categories
\newcommand*{\catFont}[1]{\mathsf{#1}} 
\newcommand*{\catVarFont}[1]{\mathcal{#1}}

\newcommand{\Set}{\catFont{Set}}
\newcommand{\Top}{\catFont{Top}}
\newcommand{\Cat}{\catFont{Cat}}
\newcommand{\Grp}{\catFont{Grp}}
\newcommand{\Mod}{\catFont{Mod}}
\newcommand{\Sub}{\catFont{Sub}}
\newcommand{\FP}{\catFont{FP}}
\newcommand{\Pos}{\catFont{Pos}}
\newcommand{\FinSet}{\catFont{FinSet}}

\newcommand{\cata}{\catVarFont{A}}
\newcommand{\catb}{\catVarFont{B}}
\newcommand{\catc}{\catVarFont{C}}
\newcommand{\catd}{\catVarFont{D}}
\newcommand{\catx}{\catVarFont{X}}
\newcommand{\caty}{\catVarFont{Y}}
\newcommand{\catZ}{\catVarFont{Z}}
%}}}

%{{{ "operators" (words in math mode)
\DeclareMathOperator{\Hom}{Hom}
\DeclareMathOperator{\End}{End}
\DeclareMathOperator{\Aut}{Aut}
\DeclareMathOperator{\im}{im}
\DeclareMathOperator{\coker}{coker}
%}}}

%{{{ arrows
\newcommand{\hookr}{\hookrightarrow}
\newcommand{\hookl}{\hookleftarrow}
\newcommand{\monor}{\rightarrowtail}
\newcommand{\monol}{\leftarrowtail}
\newcommand{\epir}{\twoheadrightarrow}
\newcommand{\epil}{\twoheadleftarrow}
\newcommand{\regepir}{\rightarrowtriangle}
\newcommand{\regepil}{\leftarrowtriangle}
%}}}

%{{{ categorical limits
\newcommand{\limr}{\varinjlim}
\newcommand{\liml}{\varprojlim}
%}}}

%{{{ analysis
\DeclareMathOperator{\dif}{d \!}
\DeclareMathOperator{\Dif}{D \!}
\newcommand{\del}{\partial}
\newcommand*{\abs}[1]{\left \lvert #1 \right \rvert}
\newcommand*{\norm}[1]{\left \lVert #1 \right \rVert}
\newcommand*{\eval}[1]{\left . #1 \right \rvert}

% partial derivative command (taken from commath package)
% usage: \pd[n]{f}{x}
\newcommand*{\pd}[3][]{\ensuremath{
\ifinner
\tfrac{\partial{^{#1}}#2}{ \partial{#3^{#1}} }
\else
\dfrac{\partial{^{#1}}#2}{ \partial{#3^{#1}} }
\fi
}}

% ordinary derivative command (taken from commath package)
% usage: \od[n]{f}{x}
\newcommand*{\od}[3][]{\ensuremath{
\ifinner
\tfrac{\dif{^{#1}}#2}{ \dif{#3^{#1}} }
\else
\dfrac{\dif{^{#1}}#2}{ \dif{#3^{#1}} }
\fi
}}

%}}}

%{{{ algebra
\newcommand{\meet}{\wedge}
\newcommand{\join}{\vee}
\newcommand{\id}{\mathrm{id}}
%}}}

%{{{ topology
\newcommand*{\interior}[1]{ {\kern0pt#1}^{\mathrm{o}} }
%}}}

%{{{ logic
\renewcommand{\diamond}{\lozenge}
\newcommand*{\denote}[1]{\llbracket #1 \rrbracket}

% \godelnum command, stolen from 
% https://www.logicmatters.net/
% 	latex-for-logicians/symbols/corner-quotes-for-godel-numbers/
\newbox\gnBoxA
\newdimen\gnCornerHgt
\setbox\gnBoxA=\hbox{$\ulcorner$}
\global\gnCornerHgt=\ht\gnBoxA
\newdimen\gnArgHgt
\def\godelnum #1{%
\setbox\gnBoxA=\hbox{$#1$}%
\gnArgHgt=\ht\gnBoxA%
\ifnum     \gnArgHgt<\gnCornerHgt \gnArgHgt=0pt%
\else \advance \gnArgHgt by -\gnCornerHgt%
\fi \raise\gnArgHgt\hbox{$\ulcorner$} \box\gnBoxA %
\raise\gnArgHgt\hbox{$\urcorner$}}

% coding function
\newcommand*{\code}[1]{\langle #1 \rangle}

%}}}

%{{{ misc symbols
\newcommand{\teq}{\triangleq}
\newcommand{\fin}{ \subseteq_{\text{fin}} }

% important words
\newcommand*{\important}[1]{\textcolor{MidnightBlue}{#1}}

% define a "danger" symbol for use when something surprising might occur
% https://tex.stackexchange.com/questions/159669/
% 	how-to-print-a-warning-sign-triangle-with-exclamation-point
% use outside of math mode!

\newcommand*{\TakeFourierOrnament}[1]{}}

%}}}

%% project specific aliases (if they exist)
\IfFileExists{../preamble.tex}{\input{../preamble.tex}}{}

%}}}

\newsavebox{\pullback}
\sbox\pullback{%
\begin{tikzpicture}%
\draw (0,0) -- (1ex,0ex);%
\draw (1ex,0ex) -- (1ex,1ex);%
\end{tikzpicture}}

%% Heading
\author{Chris Grossack}
\title{RAAG Functor Survey of Ideas}

\begin{document}
\maketitle

\section{Summary}

This whole idea came out of the recognition that the 
\important{Artin Group} construction $\Gamma \mapsto A \Gamma$ 
is actually a functor 
from the category of graphs to the category of groups. Morever,
the universal property of this functor is best formalized by writing 
$A \dashv C$ where $C$ is the \important{Commutation Graph} functor. 

For completeness, given a graph $\Gamma = (V,E)$ and a group $G$:

\[ A\Gamma \teq \langle V ~|~ [x,y] \text{ iff } x \mathrel{E} y \rangle \]

\[ CG \teq (G, \{ (g,h) ~|~ [g,h] = 1 \}) \]

Then adjointness says we have a bijection

\[ 
  \left \{ 
  \begin{tabular}{c} 
    group homomorphisms\\ 
    $A\Gamma \to G$
  \end{tabular} 
  \right \} 
  \leftrightarrow
  \left \{ 
  \begin{tabular}{c} 
    graph homomorphisms\\ 
    $\Gamma \to CG$
  \end{tabular} 
  \right \} 
\]
  
The notion of ``graph homomorphism'' is somewhat subtle, and I'll address 
exactly what is meant in the next section. I'll also be making a blog post
on exactly this topic, because I can't seem to find a table comparing various
choices of ``category of graphs'' anywhere\ldots

Regardless, adjointness endows $A$ (and $C$) with certain nice properties for
free. Most notably:

\begin{itemize}
  \item $A$ preserves colimits:
    \begin{itemize}
      \item simple example: 
        $A(\Gamma + \Delta) = A\Gamma \ast A\Delta$ 
        where $+$ (the disjoint union) is the coprudct in the category of graphs
      \item more complex example: a pushout in graphs gets sent to an 
        amalgamated free product of the corresponding groups
    \end{itemize}
  \item $C$ preserves limits:
    \begin{itemize}
      \item simple example: $C(G \times H) = CG \times CH$
      \item more complex example: 
        If $K$ is the kernel of a map $\varphi : G \to H$,
        then $CK$ is the equalizer of $C\varphi$ and $C0$ 
        in the category of graphs. (here $0$ is the constant identity map).
    \end{itemize}
  \item Since $\eta_\Gamma : \Gamma \monor CA\Gamma$, $A$ must be \emph{faithful}, 
    that is $A$ is injective on arrows.
  \item Dually, since $\epsilon_G : CAG \epir G$, $C$ must be faithful too.
\end{itemize}

Moreover, it is clear that $A$ preserves \emph{regular} monos. That is,
if $\Gamma_0$ is a full subgraph of $\Gamma$, then $A\Gamma_0$ embeds in
$A\Gamma$.

The dream is to use this machinery to better understand when 
$A\Gamma \leq A\Delta$. Such an inclusion obviously happens when 
$\Gamma$ is a full subgraph of $\Delta$, but it may occur in other situations 
as well. In fact, this adjointness says that \emph{general} homs from
$A\Gamma \to A\Delta$ are in bijection with homs $\Gamma \to CA\Delta$. 
Then we want to understand which homs $m : \Gamma \to CA\Delta$ induce 
embeddings $\widetilde{m} : A\Gamma \to A\Delta$.

Of course, there are few category-theoretic questions which are of
independent interest. I'll spend slightly less time on those.

\section{Category of Graphs}

So here's what $\mathsf{Gph}$, the \important{Category of Graphs}, maens for us:

\begin{itemize}
  \item Objects are sets $V$ equipped with a symmetric binary relation 
    $E \subseteq V \times V$. Notably, this allows self loops but 
    does not allow multi-edges.
  \item Arrows are functions which preserve this structure. That is,
    functions $f : V_1 \to V_2$ with the bonus property that 
    $x \mathrel{E_1} y \Longrightarrow fx \mathrel{E_2} fy$
\end{itemize}

It's important to allow self loops to make $A \dashv C$ work out. There
are lots of non-injective homomorphisms $A\Gamma \to G$. Notably,
since $g \in G$ commutes with itself, the map sending every generator to $g$
should be a valid hom. But in order for this to coincide with a graph hom,
we need to know that two different vertices $x$ and $y$ can both be sent to 
$g \in CG$. But if $x \mathrel{E_\Gamma} y$, then we \emph{must} have 
$g \mathrel{E_{CG}} g$ in order for the bonus property to be satisfied.

\danger \quad If we allow self loops, then $A$ can still be defined. After
all, $x$ will commute with itself anyways, so adding the relation $[x,x]$ 
won't actually impact the raag. Unfortunately, we \emph{lose} uniqueness.
Now it's possible that $A\Gamma \cong A\Delta$ but $\Gamma \not \cong \Delta$.
But they're still the same up to the removal of self-loops, and I think this
is a small price to pay for the existence of an adjoint.

Here is a quick survey of some properties of $\mathsf{Gph}$ that I suspect
might be useful.

\begin{itemize}
  \item It's ``topological over $\mathsf{Set}$''
    \begin{itemize}
      \item This means (among other things) the forgetful functor $U$ which 
        sends a graph to its set of vertices has both a left and right adjoint
        $D \dashv U \dashv K^\ell$
      \item Here $D$ sends a set to the discrete graph on that set, and $K^\ell$
        sends a set to the complete graph (with self loops!) on that set.
      \item it also means $\mathsf{Gph}$ has all limits and all colimits,
        and you can compute these by taking the (co)limits in $\mathsf{Set}$,
        and then adding the edges that you're forced to add.
    \end{itemize}
  \item Regular monos (that is, equalizers) are exactly full subgraphs. 
    This is a corollary of the statement about limits above.
  \item It's Cartesian Closed. That is, it admits a ``tensor-hom'' 
    adjunction $- \times \Gamma \dashv (-)^\Gamma$. You can think of $\Delta^\Gamma$
    as a graph which ``represents'' all of the graph homs $\Gamma \to \Delta$.
  \item It admits a (weak) subobject classifier. That is, regular monos (full subgraphs)
    $\Gamma_0 \monor \Gamma$ are in bijection with ``characteristic functions''
    $\Gamma \to \Omega$ where $\Omega$ is the following graph (which I'm too lazy to properly tikz): 
    $\circlearrowright T \-- F \circlearrowleft$. You can think of $\Omega$ as the 
    graph-theoretic analogue of the 2-element set $\{T,F\}$, which lets us
    identify a subset $A \monor X$ with the function $\chi_A : X \to \{T,F\}$.

    More formally, a (full) subgraph $\Gamma_0 \monor \Gamma$ is always a 
    pullback of the ``principal subgraph'' $\circlearrowright T$:

    \begin{center}
    \begin{tikzcd}
    \Gamma_0 \arrow[d, "m", tail] \arrow[r, "!"] \arrow[dr, phantom, "\usebox\pullback" , very near start, color=black] & \circlearrowright T \arrow[d, tail] \\
    \Gamma \arrow[r, "\chi_m"]                   & \circlearrowright T \-- F \circlearrowleft
    \end{tikzcd}
    \end{center}
\end{itemize}

All of these give us ways to create monos in the graph-theoretic world, and
have consequences for what these monos are allowed to look like. 

\section{Monads are like Burritos}

Given an adjunction $A \dashv C$, we get an endofunctor from $\mathsf{Gph}$
to itself given by the composite $CA$. Moreover, there is a canonical
arrow (the \important{unit} of the adjunction) $\eta : \Gamma \to CA\Gamma$ 
which is obtained by transposing the identity 
$\text{id} : A\Gamma \to A\Gamma \rightsquigarrow \eta : \Gamma \to CA\Gamma$.

This composite $CA$ (equipped with $\eta$)
is called the \important{Monad} associated to the adjunction $A \dashv C$.

For the briefest of brief histories, monads are one way to formalize the notion
of an algebraic structure in a category. An ``algebra'' associated to a monad
$M : \catc \to \catc$ is an object $A \in \catc$ with an arrow $\alpha : MA \to A$
so that $\alpha \eta = \text{id}$. It turns out if you look at 
the free-forgetful adjunction of groups, its algebras are exactly groups.
In general, you can view algebras of a monad as the algebras of a 
(potentially complicated) equational theory.
I think it could be interesting to see what the algebras of this adjunction are,
and it's something I'll think about more at some point. For more info, 
see Awodey's ``Category Theory'', or if you want the bible, see 
Wells and Barr's ``Toposes, Triples, and Theories'' (I haven't read it yet, fwiw).

We can also dualize this whole argument, and get a ``comonad'' 
$AC : \mathsf{Grp} \to \mathsf{Grp}$. This comes equipped with a natural
\important{counit} $\epsilon : ACG \to G$ (which is obtained by transposing the 
identity in the other direction).

Notice neither of these maps are scary. If we unwind the definitions, then
$\eta$ is the inclusion from $\Gamma$ to $CA\Gamma$ which sends a vertex $v \in \Gamma$
to the vertex $v \in CA\Gamma$. Dually, a word in $ACG$ is composed of
elements of $G$, and so there is an ``evaluation map'' $ACG \to G$ that 
simply sends each element of $G$ (viewed as a generator of $ACG$) to itself.
$\epsilon$ is exactly this evaluation map.

These are not just of theoretical interest, though. If 
$f : \Gamma \to CG$, then $\widetilde{f} : A\Gamma \to G$ is given by
$\widetilde{f} = \epsilon \circ Af$. Dually, if $g : A\Gamma \to G$,
then $\widetilde{g} : \Gamma \to CG$ is given by $\widetilde{g} = Cg \circ \eta$.

Now we see why we care: An arrow $f : \Gamma \to AC\Delta$ induces
a mono $\widetilde{f} : A\Gamma \to A\Delta$ if and only if
$\epsilon \circ Af$ is a mono. But we \emph{know} that $Af$ is a mono whenever
$f$ is a regular mono. So it suffices to check when $\epsilon \upharpoonright (Af)$
is a mono. Unfortunately, if we unwind what that means, we get a 
not-so-enlightening condition. It literally unwinds to ``$A\Gamma \to A\Delta$
is a mono''\ldots which\ldots isn't helpful.

\section{Adjunctions Induce Equivalences (and Equivalnces are Nice)}

Another approach to try (and an easy source of counterexamples) is to 
restrict attention to special subcategories. An adjunction $A \dashv C$ 
always induces an equivalence of (full) subcategories:

\[ 
  \left \{ 
    \begin{tabular}{c}
      $\Gamma$ such that\\
      $CA\Gamma \cong \Gamma$
    \end{tabular}
  \right \} 
  \simeq
  \left \{ 
    \begin{tabular}{c}
      $G$ such that\\
      $ACG \cong G$
    \end{tabular}
  \right \}
\]

So for instance, look at $K^\ell_{\aleph_0}$, the complete graph on countably
many vertices (with self loops). Then $AK^\ell_{\aleph_0}$ is a free abelian group with a countable
basis, and $CAK^\ell_{\aleph_0}$ is a graph with a vertex for each element of that 
group (of which there are countably many) and an edge connecting any two
commuting elements (which is everything). So $CAK^\ell_{\aleph_0} \cong K^\ell_{\aleph_0}$.

The cool thing about this equivalence is that it lets us 1-to-1 convert 
information between the graphs and their raags. For instance, in this setting
\emph{every} mono $m : \Gamma \to \Delta$ induces a mono 
$Am : A\Gamma \to A\Delta$. And moreover, every group-theoretic mono arises 
in this way.

The less-than-cool thing about this equivalence is that all the graphs are
infinite. After all, $CA\Gamma$ is always countable, so if we want
$\Gamma \cong CA\Gamma$, we have no choice but to look at infinite graphs.

\section{RAPL and Friends}

Since right-adjoints preserve limits, we know that a mono $m : G \to H$
induces a mono $Cm : CG \to CH$. Moreover, since every mono in $\mathsf{Grp}$
is regular, the induced mono $Cm$ is also regular. This isn't much, but
we can push it further.

Recall the kernel $K$ of a map $\varphi : G \to H$ is the following
equalizer:

\begin{center}
\begin{tikzcd}
K \arrow[r, "\iota"] & G \arrow[r, shift left=.75ex, "\varphi"] \arrow[r, shift right=.75ex, "0", swap] & H
\end{tikzcd}
\end{center}

But equalizers are limits, so hitting this with $C$ will give us an equalizer
diagram in $\mathsf{Gph}$. If we let $G = A\Gamma$, $H = A\Delta$, and $K = 1$
(since we want our kernel to be trivial), we get the following equalizer in
$\mathsf{Gph}$:

\begin{center}
\begin{tikzcd}
\circlearrowright e \arrow[r, "\iota"] & CA\Gamma \arrow[r, shift left=.75ex, "C \varphi"] \arrow[r, shift right=.75ex, "C0", swap] & CA\Delta
\end{tikzcd}
\end{center}

Here $\iota$ sends $e$ to the vertex representing the identity in $CA\Gamma$.

Conjecture: The converse is also true. That is, given an equalizer in $\mathsf{Gph}$

\begin{center}
\begin{tikzcd}
\circlearrowright e \arrow[r, "\iota"] & CA\Gamma \arrow[r, shift left=.75ex, "f"] \arrow[r, shift right=.75ex, "C0", swap] & CA\Delta
\end{tikzcd}
\end{center}

we must have $f = C\varphi$ for some $\varphi : A\Gamma \to A\Delta$ an embedding of groups.

Obviously, this could prove useful if true. This is probably where I'm going to
focus my efforts for the next little while.


\section{An Excuse to Say Big Words: QuasiTopoi and You}

Remember that long set of stuff I said about the category $\mathsf{Gph}$?
It turns out those are the defining characteristics of a \important{QuasiTopos}.
QuasiTopoi basically look like $\mathsf{Set}$, and most arguments you can
run in $\mathsf{Set}$ are also true in general quasitopoi. 

The way to make this statement precise is by defining set builder notation in your (quasi)topos. 
Then you can argue with the same naive set theory we use every day.
For instance, the initial object plays the role of the empty set,
and the terminal object plays the role of a singleton set. Then by 
looking at the exponential $\Omega^X$ we can construct the ``powerset'' of $X$.
Similarly (partial) functions $f : X \to Y$ can be represented as sets of
pairs $\{(x,y)\} \monor X \times Y$. Plus, given a logical statement
$\varphi$, we can form the ``subset'' $\{ x ~|~ \varphi(x) \} \monor X$ by
cleverly using the subobject classifier $\Omega$. Many topoi 
(including $\mathsf{Gph}$, I'm pretty sure) contain a ``natural numbers object'',
which is an object $\N$ that satisfies the induction principle inside the topos.
Once you have this, you can pick your favorite construction of the
reals, and before you know it you're doing analysis in this strange world.
Even without a natural numbers object, you can define algebraic structures
internal to this topos by considering, for instance, an object $G$ 
coupled with an arrow $m : G \times G \to G$ satisfying the group axioms.
The definition can be moved, wholesale, to the ``internal logic'' of the (quasi)topos.

Since we typically argue with set-theoretic proofs, most theorems end up 
staying true inside a (quasi)topos as well. The only place you have to be 
somewhat delicate is with nonconstructive arguments. In general, 
(quasi)topoi might fail to satisfy:

\begin{itemize}
  \item The axiom of choice
  \item Double negation elimination ($\{ x ~|~ \lnot \lnot \varphi(x) \} \neq \{ x ~|~ \varphi(x)\}$ in general)
\end{itemize}

This means you have to be somewhat careful with proofs by contradiction and
theorems that need choice. The good news is that these are really the 
\emph{only} caveats.

This is mainly of theoretical interest. It's entirely opaque to me if knowing
that $\mathsf{Gph}$ is a quasitopos will be useful (I suspect it won't be). 
But it would be extremely exciting if it \emph{does} turn out to be useful.

\section{Some Obstructions}

The extension graph $\Delta^e$ is a full subgraph of $CA\Delta$, and so
we have a regular mono $m : \Delta^e \monor CA\Delta$. I suspect it will
be easy to find monos $f$ which correspond to embeddings $A\Gamma \monor A\Delta$
which do \emph{not} factor through $m$, in the sense that no choice of dotted
arrow will make the following commute:

\begin{center}
\begin{tikzcd}
\Gamma \arrow[rd, "f"', tail] \arrow[r, dotted, tail] & \Delta^e \arrow[d, "m", tail] \\
                                                      & CA\Delta                     
\end{tikzcd}
\end{center}

\emph{But} this isn't actually helpful. Because Kim \& Koberda's claim has
nothing to do with $f$ factoring through $m$. It only asks that $\Gamma$
be a subgraph of $\Delta^e$. Asking for this factorization is asking that
$\Gamma$ be a subgraph of $\Delta^e$ in some ``natural'' or ``obvious'' way,
but it's possible that there's some random noncanonical embedding that
category-theoretic methods can't detect.

Another issue is that of the Clique graph $\Delta^e_k$. The clique graph
construction isn't obviously functorial (here our choice of $\mathsf{Gph}$ fails us,
since $k$-cliques might get squashed to a single vertex with a self loop).
I'm going to think about ways to attack this, but I'm not sure if the category
theoretic language will be directly useful in understanding the clique graph
of the extension graph. I'll code up the clique graph sometime this week, and
run some simple tests. Ideally we'll find a graph $\Lambda$ which is
a (full) subgraph of $\Delta^e_k$ but \emph{isn't} a subgraph of $CA\Delta$.
That would show that not all subgraphs of $\Delta^e_k$ induce inclusions
into $A\Delta$.

% \bibliographystyle{plain}
% \bibliography{\string~/.bib.bib}

\end{document}
