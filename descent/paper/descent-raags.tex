\documentclass[12pt]{article}
% \documentclass[10pt,xcolor={dvipsnames}]{beamer}
% \usetheme{Berlin}

%{{{ Preamble

%%%%%%%%{{{ Packages
%%%%%%%%%%%%%%%%%%%%

% the default margins have always felt big to me
% if beamer: comment this out
\usepackage[margin=1in]{geometry}

% we're gay, we like colors
% if beamer: comment this out
\usepackage[dvipsnames]{xcolor}

% we aren't animals - we might use more than ASCII
\usepackage[utf8]{inputenc}
\usepackage[T1]{fontenc}

% obligatory math environments, symbols, and theorems
\usepackage{amsmath, amssymb, amsthm}

% moar symbols
\usepackage{stmaryrd}

% obligatory citation library
\usepackage{natbib}

% sometimes you gotta draw stuff, like _c_ommutative _d_iagrams.
\usepackage{tikz, tikz-cd}

% sometimes you gotta put in pretty pictures
\usepackage{graphicx}

% sometimes you gotta write code
\usepackage{listings}

% proof trees are useful
\usepackage{proof}

% use \mathbbm for bb-numerals, use \bm for bold math symbols.
\usepackage{bbm, bm}

% I like clickable links in pdfs
\usepackage{hyperref}

% Convenient todo-notes and missing figure boxes
\usepackage{todonotes}

% remove paragraph indentation
\usepackage{parskip}

% allow fancy stuff (custom headers and footers)
\usepackage{fancyhdr}

% pretty boxes are pretty
\usepackage[framemethod=TikZ]{mdframed}

%q.uiver.app stuff
%  is necessary to draw curved arrows.
\usetikzlibrary{calc}
%  is necessary to draw squiggly arrows.
\usetikzlibrary{decorations.pathmorphing}

% A TikZ style for curved arrows of a fixed height, due to AndréC.
\tikzset{curve/.style={settings={#1},to path={(\tikztostart)
    .. controls ($(\tikztostart)!\pv{pos}!(\tikztotarget)!\pv{height}!270:(\tikztotarget)$)
    and ($(\tikztostart)!1-\pv{pos}!(\tikztotarget)!\pv{height}!270:(\tikztotarget)$)
    .. (\tikztotarget)\tikztonodes}},
    settings/.code={\tikzset{quiver/.cd,#1}
        \def\pv##1{\pgfkeysvalueof{/tikz/quiver/##1}}},
    quiver/.cd,pos/.initial=0.35,height/.initial=0}

% TikZ arrowhead/tail styles.
\tikzset{tail reversed/.code={\pgfsetarrowsstart{tikzcd to}}}
\tikzset{2tail/.code={\pgfsetarrowsstart{Implies[reversed]}}}
\tikzset{2tail reversed/.code={\pgfsetarrowsstart{Implies}}}
% TikZ arrow styles.
\tikzset{no body/.style={/tikz/dash pattern=on 0 off 1mm}}

%}}}

%%%%%%%%{{{ Formatting
%%%%%%%%%%%%%%%%%%%%%%

% prevent orphans/widows
\clubpenalty = 10000
\widowpenalty = 10000

% never break words across lines -- hyphens are stupid
\hyphenpenalty 10000
\exhyphenpenalty 10000

% forcibly overlay two symbols on top of each other
% used for the \cupdot and \bigcupdot commands
% shamelessly taken from here:
% https://tex.stackexchange.com/questions/3964/
%   mathematical-symbol-for-disjoint-set-union
\makeatletter
\def\moverlay{\mathpalette\mov@rlay}
\def\mov@rlay#1#2{\leavevmode\vtop}}

%%%%%%%%{{{ Environments
%%%%%%%%%%%%%%%%%%%%%%%%

\newtheorem{thm}{Theorem}
\newtheorem*{thm*}{Theorem}

\newtheorem*{rmk}{Remark}

% add some slashes to mark the end of a definition. 
% shamelessly stolen from 
% https://tex.stackexchange.com/
% questions/291346/marking-the-end-of-a-definition

\theoremstyle{definition}
\newtheorem{defn/}{Definition}
\newtheorem*{defn*/}{Definition}

\newcommand{\defnendsymbol}%
{%
  \mathbin{\rotatebox[origin=c]{-45}{$\parallel$}}%
}

\newenvironment{defn}
  {\renewcommand{\qedsymbol}{$\defnendsymbol$}%
   \pushQED{\qed}\begin{defn/}}
  {\popQED\end{defn/}}

\newenvironment{defn*}
  {\renewcommand{\qedsymbol}{$\defnendsymbol$}%
   \pushQED{\qed}\begin{defn*/}}
  {\popQED\end{defn*/}}

% stuff for submitting homeworks
% beamer: comment out all 3 
\theoremstyle{theorem}

\newtheorem*{problem}{Problem}

\newenvironment{soln}{\begin{proof}[Solution]}{\end{proof}}

%}}}

%%{{{ Aliases and Commands
%%%%%%%%%%%%%%%%%%%%%%%%%%

%{{{ blackboard letters
\newcommand{\N}{\mathbb{N}}
\newcommand{\Z}{\mathbb{Z}}
\newcommand{\Q}{\mathbb{Q}}
\newcommand{\R}{\mathbb{R}}
\newcommand{\C}{\mathbb{C}}
%}}}

%{{{ categories
\newcommand*{\catFont}[1]{\mathsf{#1}} 
\newcommand*{\catVarFont}[1]{\mathcal{#1}}

\newcommand{\Set}{\catFont{Set}}
\newcommand{\Top}{\catFont{Top}}
\newcommand{\Cat}{\catFont{Cat}}
\newcommand{\Grp}{\catFont{Grp}}
\newcommand{\Mod}{\catFont{Mod}}
\newcommand{\Sub}{\catFont{Sub}}
\newcommand{\FP}{\catFont{FP}}
\newcommand{\Pos}{\catFont{Pos}}
\newcommand{\FinSet}{\catFont{FinSet}}

\newcommand{\catA}{\catVarFont{A}}
\newcommand{\catB}{\catVarFont{B}}
\newcommand{\catC}{\catVarFont{C}}
\newcommand{\catD}{\catVarFont{D}}
\newcommand{\catE}{\catVarFont{E}}
\newcommand{\catX}{\catVarFont{X}}
\newcommand{\catY}{\catVarFont{Y}}
\newcommand{\catz}{\catVarFont{Z}}
%}}}

%{{{ arrows
\newcommand{\hookr}{\hookrightarrow}
\newcommand{\hookl}{\hookleftarrow}
\newcommand{\monor}{\rightarrowtail}
\newcommand{\monol}{\leftarrowtail}
\newcommand{\epir}{\twoheadrightarrow}
\newcommand{\epil}{\twoheadleftarrow}
\newcommand{\regepir}{\rightarrowtriangle}
\newcommand{\regepil}{\leftarrowtriangle}
%}}}

%{{{ categorical limits
\newcommand{\limr}{\varinjlim}
\newcommand{\liml}{\varprojlim}
%}}}

%{{{ analysis
\DeclareMathOperator{\dif}{d \!}
\DeclareMathOperator{\Dif}{D \!}
\newcommand{\del}{\partial}
\newcommand*{\abs}[1]{\left \lvert #1 \right \rvert}
\newcommand*{\norm}[1]{\left \lVert #1 \right \rVert}
\newcommand*{\eval}[1]{\left . #1 \right \rvert}

\newcommand*{\dx}{\ \dif x}
\newcommand*{\dy}{\ \dif x}
\newcommand*{\ds}{\ \dif s}
\newcommand*{\dt}{\ \dif t}

\newcommand*{\dm}{\ \dif m}
\newcommand*{\dmu}{\ \dif \mu}
\newcommand*{\dlambda}{\ \dif \lambda}

% partial derivative command (taken from commath package)
% usage: \pd[n]{f}{x}
\newcommand*{\pd}[3][]{\ensuremath{
\ifinner
\tfrac{\partial{^{#1}}#2}{ \partial{#3^{#1}} }
\else
\dfrac{\partial{^{#1}}#2}{ \partial{#3^{#1}} }
\fi
}}

% ordinary derivative command (taken from commath package)
% usage: \od[n]{f}{x}
\newcommand*{\od}[3][]{\ensuremath{
\ifinner
\tfrac{\dif{^{#1}}#2}{ \dif{#3^{#1}} }
\else
\dfrac{\dif{^{#1}}#2}{ \dif{#3^{#1}} }
\fi
}}

% restriction of a map
\newcommand*{\restrict}{\upharpoonright}

% almost everywhere
\renewcommand*{\ae}{~\mathrm{a.e.}}

% disjoint union
\newcommand{\dotcup}{\charfusion[\mathbin]{\cup}{\cdot}}
\newcommand{\bigdotcup}{\charfusion[\mathop]{\bigcup}{\cdot}}

% I can't ever remember which one it is... 
% So just have both so I can't be wrong
\newcommand{\cupdot}{\charfusion[\mathbin]{\cup}{\cdot}}
\newcommand{\bigcupdot}{\charfusion[\mathop]{\bigcup}{\cdot}}

% indicator function
% change to \chi, make it a subscript, etc. as necessary
\newcommand{\ind}[1]{\mathbbm{1}[#1]}

% epsilon alias... I really am this lazy
\newcommand{\eps}{\epsilon}

%}}}

%{{{ algebra
\newcommand{\meet}{\wedge}
\newcommand{\join}{\vee}
\newcommand{\id}{\mathrm{id}}
\newcommand{\normal}{\vartriangleleft}

\DeclareMathOperator{\Hom}{Hom}
\DeclareMathOperator{\End}{End}
\DeclareMathOperator{\Aut}{Aut}
\DeclareMathOperator{\im}{im}
\DeclareMathOperator{\coker}{coker}

\DeclareMathOperator{\Tor}{Tor}
\DeclareMathOperator{\Ext}{Ext}
%}}}

%{{{ number theory
\DeclareMathOperator{\Li}{Li}
%}}}

%{{{ topology
\newcommand*{\interior}[1]{ {\kern0pt#1}^{\mathrm{o}} }
%}}}

%{{{ lie theory

\newcommand*{\LieGrpFont}{\mathsf}

\newcommand*{\GLC}[1]{\LieGrpFont{GL}(#1,\C)}
\newcommand*{\GLR}[1]{\LieGrpFont{GL}(#1,\R)}
\newcommand*{\GLH}[1]{\LieGrpFont{GL}(#1,\mathbb{H})}

\newcommand*{\SLC}[1]{\LieGrpFont{SL}(#1,\C)}
\newcommand*{\SLR}[1]{\LieGrpFont{SL}(#1,\R)}
\newcommand*{\SLH}[1]{\LieGrpFont{SL}(#1,\mathbb{H})}

\newcommand*{\U}[1]{\LieGrpFont{U}(#1)}
\renewcommand*{\O}[1]{\LieGrpFont{O}(#1)}

\newcommand*{\SU}[1]{\LieGrpFont{SU}(#1)}
\newcommand*{\SO}[1]{\LieGrpFont{SO}(#1)}

\newcommand*{\Sp}[1]{\LieGrpFont{Sp}(#1)}

\newcommand*{\Spin}[1]{\LieGrpFont{Spin}(#1)}

%}}}

%{{{ logic
\renewcommand{\diamond}{\lozenge}
\newcommand*{\denote}[1]{\llbracket #1 \rrbracket}

% \godelnum command, stolen from 
% https://www.logicmatters.net/
% 	latex-for-logicians/symbols/corner-quotes-for-godel-numbers/
\newbox\gnBoxA
\newdimen\gnCornerHgt
\setbox\gnBoxA=\hbox{$\ulcorner$}
\global\gnCornerHgt=\ht\gnBoxA
\newdimen\gnArgHgt
\def\godelnum #1{%
\setbox\gnBoxA=\hbox{$#1$}%
\gnArgHgt=\ht\gnBoxA%
\ifnum     \gnArgHgt<\gnCornerHgt \gnArgHgt=0pt%
\else \advance \gnArgHgt by -\gnCornerHgt%
\fi \raise\gnArgHgt\hbox{$\ulcorner$} \box\gnBoxA %
\raise\gnArgHgt\hbox{$\urcorner$}}

% coding function
\newcommand*{\code}[1]{\langle #1 \rangle}

% we have \models, but for some reason not \proves?
\newcommand*{\proves}{\vdash}
\newcommand*{\forces}{\Vdash}

%}}}

%{{{ misc symbols
\newcommand{\teq}{\triangleq}
\newcommand{\fin}{ \subseteq_{\text{fin}} }

% important words
\newcommand*{\important}[1]{\textcolor{MidnightBlue}{#1}}

% define a "danger" symbol for use when something surprising might occur
% https://tex.stackexchange.com/questions/159669/
% 	how-to-print-a-warning-sign-triangle-with-exclamation-point
% use outside of math mode!

\newcommand*{\TakeFourierOrnament}[1]{{%
\fontencoding{U}\fontfamily{futs}\selectfont\char#1}}
\newcommand*{\danger}{\TakeFourierOrnament{66}}

% I always forget if this is named "danger" or "warning"...
% So just name it both so I can't be wrong
\newcommand*{\warning}{\danger}

%}}}

%}}}

%% project specific aliases (if they exist)
\IfFileExists{../preamble.tex}{\input{../preamble.tex}}{}

%}}}

%% Heading
\author{Chris Grossack\\ (they/them)}
\title{A Descent Theorem for Right Angled Artin Groups}

\begin{document}
\maketitle

\begin{abstract}
  We can associate to each graph $G$ a group $AG$, called a
  \emph{right angled artin group}, freely generated by the vertices
  of $G$ with the relation that two vertices commute if and only if 
  they're adjacent in $G$. This association is functorial, so that 
  every graph homomorphism induces a group homomorphism. 
  A natural question is to try characterize which group homomorphisms
  arise in this way, and in this paper we provide an effective 
  method for detecting these homomorphisms. We do this by applying the
  methodology of \emph{comonadic descent} to the category of groups,
  and thus find the category of graphs as a certain subcategory.
\end{abstract}

\section{Introduction}
    For us, a \important{Graph} $G$ with underlying vertex set $V$ is a 
    symmetric, relation on $V$. A \important{Graph Homomorphism} between 
    a graph $G$ (with vertex set $V$) and $H$ (with vertex set $W$) is a 
    function $f : V \to W$ so that $(v_1, v_2) \in G \implies (f v_1, f v_2) \in H$.
    These obviously assemble into a category, which we call $\mathsf{Gph}$.

    Given a graph $G$ with vertex set $V$, we can form a group $AG$, the 
    \important{Right Angled Artin Group} (Raag) associated to $G$, defined as

    \[ AG \teq \langle v \in V \mid [v_1, v_2] = 1 \text{ whenever } (v_1,v_2) \in G \rangle .\]

    For example, if $G$ is a complete graph on $n$ vertices then 
    $AG \cong \mathbb{Z}^n$, and if $G$ is a discrete graph on $n$ vertices
    $AH \cong \mathbb{F}_n$ is a free group on $n$ generators. 

    In this sense, raags allow us to \emph{interpolate} between free and free
    abelian groups, and have been important objects of study in combinatorics
    and group theory for this (and other) reasons \todo{find some citations for this}. 
    Moreover, the interplay between the combinatorics of $G$ and the 
    algebra of $AG$ is remarkably intricate. See, for instance 
    \todo{more citations. Importantly, the paper on expanders and raags}.

    More recently, raags have been of interest to geometric group theorists,
    both for their geometric actions 
    \todo{cite raags acting on cubes\ldots 
    are there papers detailing some nice properties of raags? Maybe kim+koberda
    or another survey? Iirc they generalize other geometrically interesting families of groups}
    and their relationship to the curve graphs of mapping class groups
    \todo{cite a whole slew of papers here\ldots Say the word ``pseudo-anasov'', if you want to sound fancy}.

    \todo{rip off paragraph 2 of minasyan's paper}

    Subgroups of raags are of particular interest \todo{Ask Matt why}
    \todo{rip off section 2.2 of Charney's survey}

    The precise relationship between a graph and its raag is still somewhat
    mysterious, and much work has gone towards understanding \todo{etc}

    Given a group $\Gamma$, it's natural to ask if it is $AG$ for some 
    graph $G$. Moreover, given a group homomorphism $\varphi : AG \to AH$, it's 
    natural to ask when $\varphi = Af$ for some group homomorphism $f : G \to H$.

    \todo{go through and reread the papers about this. I forget exactly what's known}

    Conversely, given a group $\Gamma$ we can take its \important{Commutation Graph}
    $C\Gamma$ whose vertices are elements of $\Gamma$ and we say 
    $(\gamma_1, \gamma_2) \in C\Gamma \iff [\gamma_1, \gamma_2] = 1$.
    These graphs have been of interest to combinatorialists 
    \todo{find some citations. Definitely look at old posts on Peter Cameron's blog}
    for decades, because of \todo{reasons}

    It's not hard to see that $C : \mathsf{Grp} \to \mathsf{Gph}$ is also a 
    functor, and indeed is right adjoint to $A : \mathsf{Gph} \to \mathsf{Grp}$
    (this has been noticed by many others \todo{citations}, and merely 
    expresses the usual universal property of raags)

    \todo{have we said what we're planning to prove? Put a statement of the main theorem here in the intro}

\section{An Instructive Example}

\todo{ give an example application of the main theorem, show that it's understandable even without CT }

\section{A Brief Review of Comonadic Descent}

\todo{add a bunch of citations for descent literature}

Recall that an adjunction
$(L : \catC \to \catD) \dashv (R : \catD \to \catC)$
induces a \emph{monad} $RL : \catC \to \catC$ and a \emph{comonad}
$LR : \catD \to \catD$.

By a comonad, we mean a functor $W : \catD \to \catD$ equipped with natural
transformations $\epsilon : W \Rightarrow 1_\catD$ and $\delta : W \Rightarrow WW$
so that the following diagrams commute:

\[
    \begin{tikzcd}
    & W &&& W & WW \\
    W & WW & W && WW & WWW
    \arrow["{1_W \cdot \epsilon}", Rightarrow, from=2-2, to=2-1]
    \arrow["{\epsilon \cdot 1_W}"', Rightarrow, from=2-2, to=2-3]
    \arrow["\delta", Rightarrow, from=1-2, to=2-2]
    \arrow["{1_W}"', Rightarrow, from=1-2, to=2-1]
    \arrow["{1_W}", Rightarrow, from=1-2, to=2-3]
    \arrow["\delta"', Rightarrow, from=1-5, to=2-5]
    \arrow["\delta", Rightarrow, from=1-5, to=1-6]
    \arrow["{\delta \cdot 1_W}"', Rightarrow, from=2-5, to=2-6]
    \arrow["{1_W \cdot \delta}", Rightarrow, from=1-6, to=2-6]
    \end{tikzcd}
\]

Monads and comonads find application in settings as varied as universal algebra 
\todo{cite something on monads and something on comonads}, probability theory \todo{cite leinster's paper on ct probability}
computer science \todo{ cite Moggi's paper and some paper on comonads in CS },
and phyiscs \todo{ is this true? if so, cite it }.
Relevant for us is the theory of \important{comonadic descent}, 
which comes from algebraic geometry \todo{can we find a survey paper on descent in AG?}

A \important{$W$-coalgebra} is an object
$A \in \catD$ equipped with an arrow $\alpha : A \to WA$ so that the
diagrams in figure \ref{coalg} commute. A \important{$W$-cohomomorphism}
between coalgebras $(A,\alpha)$ and $(B,\beta)$ is an arrow $f : A \to B$
in $\mathcal{D}$ compatible with $\alpha$ and $\beta$, in the sense that 
figure \ref{cohom} commutes. When $W$ is clear from context, we simply call these
coalgebras and cohomomorphisms, and they assemble into a category $\catD_W$.

\begin{figure}
    \caption{The defining diagrams for a coalgebra}
    \label{coalg}
    \[
        \begin{tikzcd}
        A & WA && A & WA \\
        & A && WA & WWA
        \arrow["\alpha", from=1-1, to=1-2]
        \arrow["{\epsilon_A}", from=1-2, to=2-2]
        \arrow["{1_A}"', from=1-1, to=2-2]
        \arrow["\alpha"', from=1-4, to=2-4]
        \arrow["\alpha", from=1-4, to=1-5]
        \arrow["\delta"', from=2-4, to=2-5]
        \arrow["W\alpha", from=1-5, to=2-5]
        \end{tikzcd}
    \]
\end{figure}

\begin{figure}
    \caption{The defining diagram for a cohomomorphism}
    \label{cohom}
    \[
        \begin{tikzcd}
        A & B \\
        WA & WB
        \arrow["\alpha", from=1-1, to=2-1]
        \arrow["\beta", from=1-2, to=2-2]
        \arrow["f", from=1-1, to=1-2]
        \arrow["Wf"', from=2-1, to=2-2]
        \end{tikzcd}
    \]    
\end{figure}


\todo{end with the theorem saying how comonadic descent works}

\section{Applying Descent to Raags}

\section{Can we Really Compute These?}

\section{Conclusion}

\todo{ this comonad was useful, and it stands to reason the monad should be 
useful too. In fact, this seems to be related to the extension graph of\ldots
Kim and Koberda? Add a citation }

\newpage
\nocite{*}
\bibliographystyle{plain}
\bibliography{bib.bib}

\end{document}
