\documentclass[12pt]{article}
% \documentclass[10pt,xcolor={dvipsnames}]{beamer}
% \usetheme{Berlin}

%{{{ Preamble

%%%%%%%%{{{ Packages
%%%%%%%%%%%%%%%%%%%%

% the default margins have always felt big to me
% if beamer: comment this out
\usepackage[margin=1in]{geometry}

% we're gay, we like colors
% if beamer: comment this out
\usepackage[dvipsnames]{xcolor}

% we aren't animals - we might use more than ASCII
\usepackage[utf8]{inputenc}
\usepackage[T1]{fontenc}

% obligatory math environments, symbols, and theorems
\usepackage{amsmath, amssymb, amsthm}

% moar symbols
\usepackage{stmaryrd}

% obligatory citation library
\usepackage{natbib}

% sometimes you gotta draw stuff, like _c_ommutative _d_iagrams.
\usepackage{tikz, tikz-cd}

% sometimes you gotta put in pretty pictures
\usepackage{graphicx}

% sometimes you gotta write code
\usepackage{listings}

% proof trees are useful
\usepackage{proof}

% use \mathbbm for bb-numerals, use \bm for bold math symbols.
\usepackage{bbm, bm}

% I like clickable links in pdfs
\usepackage{hyperref}

% Convenient todo-notes and missing figure boxes
\usepackage{todonotes}

% remove paragraph indentation
\usepackage{parskip}

% allow fancy stuff (custom headers and footers)
\usepackage{fancyhdr}

% pretty boxes are pretty
\usepackage[framemethod=TikZ]{mdframed}

%q.uiver.app stuff
%  is necessary to draw curved arrows.
\usetikzlibrary{calc}
%  is necessary to draw squiggly arrows.
\usetikzlibrary{decorations.pathmorphing}

% A TikZ style for curved arrows of a fixed height, due to AndréC.
\tikzset{curve/.style={settings={#1},to path={(\tikztostart)
    .. controls ($(\tikztostart)!\pv{pos}!(\tikztotarget)!\pv{height}!270:(\tikztotarget)$)
    and ($(\tikztostart)!1-\pv{pos}!(\tikztotarget)!\pv{height}!270:(\tikztotarget)$)
    .. (\tikztotarget)\tikztonodes}},
    settings/.code={\tikzset{quiver/.cd,#1}
        \def\pv##1{\pgfkeysvalueof{/tikz/quiver/##1}}},
    quiver/.cd,pos/.initial=0.35,height/.initial=0}

% TikZ arrowhead/tail styles.
\tikzset{tail reversed/.code={\pgfsetarrowsstart{tikzcd to}}}
\tikzset{2tail/.code={\pgfsetarrowsstart{Implies[reversed]}}}
\tikzset{2tail reversed/.code={\pgfsetarrowsstart{Implies}}}
% TikZ arrow styles.
\tikzset{no body/.style={/tikz/dash pattern=on 0 off 1mm}}

%}}}

%%%%%%%%{{{ Formatting
%%%%%%%%%%%%%%%%%%%%%%

% prevent orphans/widows
\clubpenalty = 10000
\widowpenalty = 10000

% never break words across lines -- hyphens are stupid
\hyphenpenalty 10000
\exhyphenpenalty 10000

% forcibly overlay two symbols on top of each other
% used for the \cupdot and \bigcupdot commands
% shamelessly taken from here:
% https://tex.stackexchange.com/questions/3964/
%   mathematical-symbol-for-disjoint-set-union
\makeatletter
\def\moverlay{\mathpalette\mov@rlay}
\def\mov@rlay#1#2{\leavevmode\vtop}}

%%%%%%%%{{{ Environments
%%%%%%%%%%%%%%%%%%%%%%%%

\newtheorem{thm}{Theorem}
\newtheorem*{thm*}{Theorem}

\newtheorem*{rmk}{Remark}

% add some slashes to mark the end of a definition. 
% shamelessly stolen from 
% https://tex.stackexchange.com/
% questions/291346/marking-the-end-of-a-definition

\theoremstyle{definition}
\newtheorem{defn/}{Definition}
\newtheorem*{defn*/}{Definition}

\newcommand{\defnendsymbol}%
{%
  \mathbin{\rotatebox[origin=c]{-45}{$\parallel$}}%
}

\newenvironment{defn}
  {\renewcommand{\qedsymbol}{$\defnendsymbol$}%
   \pushQED{\qed}\begin{defn/}}
  {\popQED\end{defn/}}

\newenvironment{defn*}
  {\renewcommand{\qedsymbol}{$\defnendsymbol$}%
   \pushQED{\qed}\begin{defn*/}}
  {\popQED\end{defn*/}}

% stuff for submitting homeworks
% beamer: comment out all 3 
\theoremstyle{theorem}

\newtheorem*{problem}{Problem}

\newenvironment{soln}{\begin{proof}[Solution]}{\end{proof}}

%}}}

%%{{{ Aliases and Commands
%%%%%%%%%%%%%%%%%%%%%%%%%%

%{{{ blackboard letters
\newcommand{\N}{\mathbb{N}}
\newcommand{\Z}{\mathbb{Z}}
\newcommand{\Q}{\mathbb{Q}}
\newcommand{\R}{\mathbb{R}}
\newcommand{\C}{\mathbb{C}}
%}}}

%{{{ categories
\newcommand*{\catFont}[1]{\mathsf{#1}} 
\newcommand*{\catVarFont}[1]{\mathcal{#1}}

\newcommand{\Set}{\catFont{Set}}
\newcommand{\Top}{\catFont{Top}}
\newcommand{\Cat}{\catFont{Cat}}
\newcommand{\Grp}{\catFont{Grp}}
\newcommand{\Mod}{\catFont{Mod}}
\newcommand{\Sub}{\catFont{Sub}}
\newcommand{\FP}{\catFont{FP}}
\newcommand{\Pos}{\catFont{Pos}}
\newcommand{\FinSet}{\catFont{FinSet}}

\newcommand{\cata}{\catVarFont{A}}
\newcommand{\catb}{\catVarFont{B}}
\newcommand{\catc}{\catVarFont{C}}
\newcommand{\catd}{\catVarFont{D}}
\newcommand{\catx}{\catVarFont{X}}
\newcommand{\caty}{\catVarFont{Y}}
\newcommand{\catZ}{\catVarFont{Z}}
%}}}

%{{{ arrows
\newcommand{\hookr}{\hookrightarrow}
\newcommand{\hookl}{\hookleftarrow}
\newcommand{\monor}{\rightarrowtail}
\newcommand{\monol}{\leftarrowtail}
\newcommand{\epir}{\twoheadrightarrow}
\newcommand{\epil}{\twoheadleftarrow}
\newcommand{\regepir}{\rightarrowtriangle}
\newcommand{\regepil}{\leftarrowtriangle}
%}}}

%{{{ categorical limits
\newcommand{\limr}{\varinjlim}
\newcommand{\liml}{\varprojlim}
%}}}

%{{{ analysis
\DeclareMathOperator{\dif}{d \!}
\DeclareMathOperator{\Dif}{D \!}
\newcommand{\del}{\partial}
\newcommand*{\abs}[1]{\left \lvert #1 \right \rvert}
\newcommand*{\norm}[1]{\left \lVert #1 \right \rVert}
\newcommand*{\eval}[1]{\left . #1 \right \rvert}

\newcommand*{\dx}{\ \dif x}
\newcommand*{\dy}{\ \dif x}
\newcommand*{\ds}{\ \dif s}
\newcommand*{\dt}{\ \dif t}

\newcommand*{\dm}{\ \dif m}
\newcommand*{\dmu}{\ \dif \mu}
\newcommand*{\dlambda}{\ \dif \lambda}

% partial derivative command (taken from commath package)
% usage: \pd[n]{f}{x}
\newcommand*{\pd}[3][]{\ensuremath{
\ifinner
\tfrac{\partial{^{#1}}#2}{ \partial{#3^{#1}} }
\else
\dfrac{\partial{^{#1}}#2}{ \partial{#3^{#1}} }
\fi
}}

% ordinary derivative command (taken from commath package)
% usage: \od[n]{f}{x}
\newcommand*{\od}[3][]{\ensuremath{
\ifinner
\tfrac{\dif{^{#1}}#2}{ \dif{#3^{#1}} }
\else
\dfrac{\dif{^{#1}}#2}{ \dif{#3^{#1}} }
\fi
}}

% restriction of a map
\newcommand*{\restrict}{\upharpoonright}

% almost everywhere
\renewcommand*{\ae}{~\mathrm{a.e.}}

% disjoint union
\newcommand{\dotcup}{\charfusion[\mathbin]{\cup}{\cdot}}
\newcommand{\bigdotcup}{\charfusion[\mathop]{\bigcup}{\cdot}}

% I can't ever remember which one it is... 
% So just have both so I can't be wrong
\newcommand{\cupdot}{\charfusion[\mathbin]{\cup}{\cdot}}
\newcommand{\bigcupdot}{\charfusion[\mathop]{\bigcup}{\cdot}}

% indicator function
% change to \chi, make it a subscript, etc. as necessary
\newcommand{\ind}[1]{\mathbbm{1}[#1]}

% epsilon alias... I really am this lazy
\newcommand{\eps}{\epsilon}

%}}}

%{{{ algebra
\newcommand{\meet}{\wedge}
\newcommand{\join}{\vee}
\newcommand{\id}{\mathrm{id}}
\newcommand{\normal}{\vartriangleleft}

\DeclareMathOperator{\Hom}{Hom}
\DeclareMathOperator{\End}{End}
\DeclareMathOperator{\Aut}{Aut}
\DeclareMathOperator{\im}{im}
\DeclareMathOperator{\coker}{coker}

\DeclareMathOperator{\Tor}{Tor}
\DeclareMathOperator{\Ext}{Ext}
%}}}

%{{{ number theory
\DeclareMathOperator{\Li}{Li}
%}}}

%{{{ topology
\newcommand*{\interior}[1]{ {\kern0pt#1}^{\mathrm{o}} }
%}}}

%{{{ lie theory

\newcommand*{\LieGrpFont}{\mathsf}

\newcommand*{\GLC}[1]{\LieGrpFont{GL}(#1,\C)}
\newcommand*{\GLR}[1]{\LieGrpFont{GL}(#1,\R)}
\newcommand*{\GLH}[1]{\LieGrpFont{GL}(#1,\mathbb{H})}

\newcommand*{\SLC}[1]{\LieGrpFont{SL}(#1,\C)}
\newcommand*{\SLR}[1]{\LieGrpFont{SL}(#1,\R)}
\newcommand*{\SLH}[1]{\LieGrpFont{SL}(#1,\mathbb{H})}

\newcommand*{\U}[1]{\LieGrpFont{U}(#1)}
\renewcommand*{\O}[1]{\LieGrpFont{O}(#1)}

\newcommand*{\SU}[1]{\LieGrpFont{SU}(#1)}
\newcommand*{\SO}[1]{\LieGrpFont{SO}(#1)}

\newcommand*{\Sp}[1]{\LieGrpFont{Sp}(#1)}

\newcommand*{\Spin}[1]{\LieGrpFont{Spin}(#1)}

%}}}

%{{{ logic
\renewcommand{\diamond}{\lozenge}
\newcommand*{\denote}[1]{\llbracket #1 \rrbracket}

% \godelnum command, stolen from 
% https://www.logicmatters.net/
% 	latex-for-logicians/symbols/corner-quotes-for-godel-numbers/
\newbox\gnBoxA
\newdimen\gnCornerHgt
\setbox\gnBoxA=\hbox{$\ulcorner$}
\global\gnCornerHgt=\ht\gnBoxA
\newdimen\gnArgHgt
\def\godelnum #1{%
\setbox\gnBoxA=\hbox{$#1$}%
\gnArgHgt=\ht\gnBoxA%
\ifnum     \gnArgHgt<\gnCornerHgt \gnArgHgt=0pt%
\else \advance \gnArgHgt by -\gnCornerHgt%
\fi \raise\gnArgHgt\hbox{$\ulcorner$} \box\gnBoxA %
\raise\gnArgHgt\hbox{$\urcorner$}}

% coding function
\newcommand*{\code}[1]{\langle #1 \rangle}

% we have \models, but for some reason not \proves?
\newcommand*{\proves}{\vdash}
\newcommand*{\forces}{\Vdash}

%}}}

%{{{ misc symbols
\newcommand{\teq}{\triangleq}
\newcommand{\fin}{ \subseteq_{\text{fin}} }

% important words
\newcommand*{\important}[1]{\textcolor{MidnightBlue}{#1}}

% define a "danger" symbol for use when something surprising might occur
% https://tex.stackexchange.com/questions/159669/
% 	how-to-print-a-warning-sign-triangle-with-exclamation-point
% use outside of math mode!

\newcommand*{\TakeFourierOrnament}[1]{{%
\fontencoding{U}\fontfamily{futs}\selectfont\char#1}}
\newcommand*{\danger}{\TakeFourierOrnament{66}}

% I always forget if this is named "danger" or "warning"...
% So just name it both so I can't be wrong
\newcommand*{\warning}{\danger}

%}}}

%}}}

%% project specific aliases (if they exist)
\IfFileExists{../preamble.tex}{\input{../preamble.tex}}{}

%}}}

%% Heading
\author{Chris Grossack\\ (they/them)}
\title{Descent For RAAGs}

\begin{document}
\maketitle

\section{Some Definitions}

Let $\mathsf{Gph}$ be the category of reflexive graphs. That is, sets $V$ 
equipped with a reflexive symmetric relation $E \subseteq V \times V$, which
we think of as simple graphs with a self loop at each vertex. The arrows of
this category are functions $f : V_1 \to V_2$ so that $(v,w) \in E_1$ implies
$(fv,fw) \in E_2$. We use letters like $\Gamma$, $\Delta$, etc. to denote 
graphs, since letters like $G$ and $H$ are reserved for groups. Similarly,
we will use greek leters for graph homomorphisms and roman letters for
group homomorphisms.

Let $A : \mathsf{Gph} \to \mathsf{Grp}$ be the functor sending a graph to its
right angled artin group, and $C : \mathsf{Grp} \to \mathsf{Gph}$ be the functor
sending a group to its "commutation graph". The graph whose vertices are the 
elements of $G$, where $g_1$ and $g_2$ are related if and only if $g_1g_2 = g_2g_1$.

It's easy to verify that $A \dashv C$.

\section{A Natural Question}

Given a group homomorphism $f : A\Gamma \to A\Delta$, when is $f = A \varphi$
for a graph homomorphism $\varphi : \Gamma \to \Delta$?

The answer, I'm pretty sure, is this:

$f = A \varphi$ if and only if $f$ satisfies the following \important{descent condition}:

\[\begin{tikzcd}
	A\Gamma && A\Delta \\
	\\
	ACA\Gamma && ACA\Delta
	\arrow["f", from=1-1, to=1-3]
	\arrow["ACf", from=3-1, to=3-3]
	\arrow["{A \eta}"{description}, from=1-1, to=3-1]
	\arrow["A\eta"{description}, from=1-3, to=3-3]
\end{tikzcd}\]

Notice this diagram is \emph{not} always commutative since the vertical arrows
are $A \eta$, which is not the functor we applied to $f$ in the bottom of the square.

Concretely, each graph $\Gamma$ naturally embeds into its 
\important{monad graph} $CA\Gamma$, and this embedding is 
$\eta : \Gamma \to CA\Gamma$, which sends $\gamma \mapsto \gamma$ 
(where the $\gamma$ in the codomain is to be thought of as the vertex 
associated to the word ``$\gamma$'' of length $1$ in $A\Gamma$).

The vertical arrows are the homomorphisms $A\Gamma \to ACA\Gamma$ (resp. $\Delta$) 
induced on RAAGs by the graph embeddings $\eta$.

Notice this is obviously a \emph{necessary} condition, since if $f = A\varphi$
the square becomes

\[\begin{tikzcd}
	A\Gamma && A\Delta \\
	\\
	ACA\Gamma && ACA\Delta
	\arrow["A\varphi", from=1-1, to=1-3]
	\arrow["ACA\varphi", from=3-1, to=3-3]
	\arrow["{A \eta}"{description}, from=1-1, to=3-1]
	\arrow["A\eta"{description}, from=1-3, to=3-3]
\end{tikzcd}\]

which \emph{does} obviously commute, since now the vertical arrows are 
$ACA$, the same functor we're applying to $\varphi$.


\section{An Example}

Let's look at the simplest possible case:

Let $\Gamma = \{ v \}$ and $\Delta = \{ w \}$ be two one-vertex graphs.
Then $A \Gamma = \langle v \rangle$ and $A \Delta = \langle w \rangle$,
and we want to detect when a homomorphism between these groups came from a
homomorphism of their underlying graphs.

Well, $CA \Gamma$ is the complete graph on $\mathbb{Z}$ many vertices, each
representing one of the $v^n \in A \Gamma$. 
Then $\eta : \Gamma \to CA \Gamma$ sends $v \mapsto v^1$ in this graph,
and the construction for $\Delta$ is the same.

As a warning, notice that $ACA\Gamma = \langle v^n \mid [v^n, v^m] = 1 \rangle$
is generated by \emph{formal symbols} named $v^n$. In particular,
$v^2 v^3 \neq v^5$ in this group. The first is a word of length $2$, the latter
is a word of length $1$, and there are no relations saying we can combine
the symbols.

\bigskip

Let's first look at a map that \emph{does} come from a graph homomorphism.
Say, $f : \langle v \rangle \to \langle w \rangle$ given by $fv = w$.

Then we want to consult the square

\[\begin{tikzcd}
	{\langle v \rangle} && {\langle w \rangle} \\
	\\
	{\langle v^n \mid [v^n,v^m] = 1 \rangle} && {\langle w^n \mid [w^n, w^m] = 1 \rangle}
	\arrow["{v \mapsto w}", from=1-1, to=1-3]
	\arrow["{v^n \mapsto w^n}", from=3-1, to=3-3]
	\arrow["{v \mapsto v^1}"{description}, from=1-1, to=3-1]
	\arrow["{w \mapsto w^1}"{description}, from=1-3, to=3-3]
\end{tikzcd}\]

which clearly does commute. This tells us that $f$ comes from a graph homomorphism,
as indeed it did!

\bigskip

Next, let's look at a map which \emph{doesn't} come from a graph homomorphism.
Say, $g : \langle v \rangle \to \langle w \rangle$ given by $gv = w^2$.

Now our square is

\[\begin{tikzcd}
	{\langle v \rangle} && {\langle w \rangle} \\
	\\
	{\langle v^n \mid [v^n,v^m] = 1 \rangle} && {\langle w^n \mid [w^n, w^m] = 1 \rangle}
	\arrow["{v \mapsto w^2}", from=1-1, to=1-3]
	\arrow["{v^n \mapsto w^{2n}}", from=3-1, to=3-3]
	\arrow["{v \mapsto v^1}"{description}, from=1-1, to=3-1]
	\arrow["{w \mapsto w^1}"{description}, from=1-3, to=3-3]
\end{tikzcd}\]

which does \emph{not} commute (even though it seems to at first glance). 
Indeed, if we chase the image of $v$ around the top right of the square, then
(remembering righthand map is $w \mapsto w^1$, where $w^1$ is a formal symbol):

\[ v \mapsto w^2 \mapsto (w^1)^2 \]

If instead we chase around the lower left of the square, we get:

\[ v \mapsto v^1 \mapsto w^2 \]

since $(w^1)^2 \neq w^2$ in this group, we have successfully detected that 
$f$ did \emph{not} come from a graph homomorphism!

\section{Descent}

There's probably a more direct way to prove this now that we know the right
condition, but I went through abstract nonsense, since that's what inspired
me to think about this in the first place.

The key takeaway is this: We have an adjunction

\[\begin{tikzcd}
	{\mathsf{Grp}} \\
	\\
	{\mathsf{Gph}}
	\arrow[""{name=0, anchor=center, inner sep=0}, "A", curve={height=-12pt}, from=3-1, to=1-1]
	\arrow[""{name=1, anchor=center, inner sep=0}, "C", curve={height=-12pt}, from=1-1, to=3-1]
	\arrow["\dashv"{anchor=center}, draw=none, from=0, to=1]
\end{tikzcd}\]

and we want to understand the image of $A$ in $\mathsf{Grp}$. There's a lot
of literature dedicated to this subject, called \important{Comonadic Descent}.
I was reading about this stuff for personal interest last night, and once I 
actually understood it\footnote{inasfar as I understand anything more complicated than linear algebra}, 
I realized it's perfectly situated to solve this RAAG problem I was thinking about
all those years ago!

So then, what's the idea? Well just like we get a \emph{monad} 
$CA : \mathsf{Gph} \to \mathsf{Gph}$ (which is where the monad graph comes from),
we dually get a \emph{comonad} $AC : \mathsf{Grp} \to \mathsf{Grp}$. The 
coalgebras of this comonad are groups $G$ equipped with a homomorphism 
$f_G : G \to ACG$, and the cohomomorphisms between coalgebras are group 
homomorphisms $k : G \to H$ making the obvious diagram commute:

\[\begin{tikzcd}
	G && H \\
	\\
	ACG && ACH
	\arrow["{f_G}", from=1-1, to=3-1]
	\arrow["{f_H}", from=1-3, to=3-3]
	\arrow["k"{description}, from=1-1, to=1-3]
	\arrow["ACk"{description}, from=3-1, to=3-3]
\end{tikzcd}\]

Next, it turns out that each $A\Gamma$ comes naturally equipped with a 
coalgebra structure. Namely the one given by  $A \eta : A\Gamma \to ACA\Gamma$.

This means the functor $A : \mathsf{Gph} \to \mathsf{Grp}$ factors through the
category of coalgebras and cohomomorphisms. Wouldn't it be nice if this were an
equivalence of categories?
Then we would know that every coalgebra is (up to isomorphism) of the form 
$A\Gamma$, and moreover we would know that cohomomorphisms are \emph{exactly} 
the group homs that come from a graph hom!

In fact, unless I've made a mistake, this is exactly the situation we find 
ourselves in. We need to do a \emph{little} bit of work, though. It can't be
category theory all the way down.

There's a theorem (due to Beck) that says $A$ is an equivalence of categories
between $\mathsf{Gph}$ and the coalgebras in $\mathsf{Grp}$ if and only if

\begin{enumerate}
  \item $A$ reflects isomorphisms (that is, $A\Gamma \cong A\Delta$ if and only if $\Gamma \cong \Delta$)
  \item $\mathsf{Gph}$ has, and $A$ preserves, ``equalizers of coreflexive pairs'' (we'll get to this one)
\end{enumerate}

Now, condition $1$ is definitely satisfied here. This is theorem $2.7$ from
Koberda's RAAG Course \href{https://users.math.yale.edu/users/koberda/raagcourse.pdf}{here}.

Condition $2$ is off to a good start. $\mathsf{Gph}$ is complete and cocomplete,
so whatever ``coreflexive pairs'' are, they'll have equalizers.

But $A$ definitely \emph{doesn't} preserve general equalizers. So we're going to 
have to actually look into what ``coreflexive pairs'' are, and we'll have 
to check by hand that $A$ really does preserve their equalizers.

\section{Equalizers of Coreflexive Pairs}

A \important{coreflexive pair} is a pair of arrows with a common retract. 
A situation

\[\begin{tikzcd}
	\Gamma && \Delta
	\arrow["\alpha", shift left=3, from=1-1, to=1-3]
	\arrow["\beta"', shift right=3, from=1-1, to=1-3]
	\arrow["\rho"{description}, from=1-3, to=1-1]
\end{tikzcd}\]

where $\rho \alpha = 1_\Gamma = \rho \beta$.

Now, we want to show that if $\Theta$ is the equalizer of $\alpha$ and $\beta$,
as computed in $\mathsf{Gph}$, 
then $A\Theta$ should still be the equalizer of $A \alpha$ and $A \beta$, 
as computed in $\mathsf{Grp}$.

Well thankfully, $\Theta$ is pretty easy to understand. It's the full subgraph
of $\Gamma$ on those vertices where $\alpha v = \beta v$.

So then $A\Theta = \langle v \mid \alpha v = \beta v \rangle \leq A \Gamma$.

What about the equalizer as computed in $\mathsf{Grp}$? Well, that's also not
so bad. It's the group $G = \{ g \mid \alpha g = \beta g \} \leq A\Gamma$.

Since $A\Theta$ is obviously a subgroup of $G$, all we need to do is show that 
$G$ is a subgroup of $A\Theta$! 

That is, we need to show that when 
$\alpha$ and $\beta$ admit a common retract $\rho$,
any group element $g$ with $\alpha g = \beta g$ should be a word in the
vertices $v$ with $\alpha v =  \beta v$.

\section{The Proof}

Sorry if this is hard to read\ldots I don't have any experience at all writing
these proofs via combinatorics on words. I'm actually not \emph{fully} convinced
of the proof as written either, but I think I have convinced myself that the
result is true. I would love to meet sometime to figure out how to nail this down,
if you have the time.

\bigskip

First, since $\rho \alpha = 1_\Gamma = \rho \beta$, we know that $\alpha$ and
$\beta$ are both strongly injective on vertices. By this we mean that not 
only are $\alpha$ and $\beta$ injective, but moreover if
$\alpha v = \beta w$, then $v = w$.

Next, since $\rho$ is a graph
hom, we see that $v$ and $w$ are $\Gamma$-related if and only if $\alpha v$
and $\alpha w$ (resp. $\beta v$ and $\beta w$) are $\Delta$-related. Thus 
$v$ and $w$ commute in $A \Gamma$ if and only if their images under $\alpha$
(resp. $\beta$) commute in $A \Delta$.

Now, in Charney's \emph{An Introduction to RAAGs} 
(\href{https://people.brandeis.edu/~charney/papers/RAAGfinal.pdf}{here}),
section $2.3$ ``The Word Problem'' shows that for each $g \in A\Gamma$,
there's a minimal length word $g = w_0 w_1 \ldots w_k$ so that each 
$w_i$ is a word in the vertices of $\Gamma$, and each $w_i$ is of maximal 
length so that the letters in some fixed $w_i$ mutually commute.

Moreover, this decomposition $g = w_0 w_1 \ldots w_k$ is unique up to 
commuting the letters in each fixed $w_i$.

So now, let's say that $\alpha g = \beta g$. Fix such a decomposition
$g = w_0 w_1 \ldots w_k$, and look at 

\[ (\alpha w_0) (\alpha w_1) \ldots (\alpha w_k) = (\beta w_0) (\beta w_1) \ldots (\beta w_k) \]

these representations of $\alpha g = \beta g$ are both minimal length, as 
we could hit a shorter representation with $\rho$ in order to get a 
shorter representation for $g$. So then, by the uniqueness of canonical form 
described in Charney's paper, 
we see that each $\alpha w_i$ and $\beta w_i$ are equal up to permuting the 
letters in each. 

So we restrict attention to each 
$w_i = \gamma_1^{n_1} \gamma_2^{n_2} \ldots \gamma_k^{n_k}$ separately, say

\[ 
  (\alpha \gamma_1^{n_1}) (\alpha \gamma_2^{n_2}) \ldots (\alpha \gamma_k^{n_k}) = 
  \delta_1^{n_1} \delta_2^{n_2} \ldots \delta_k^{n_k} =
  (\beta \gamma_1^{n_1}) (\beta \gamma_2^{n_2}) \ldots (\beta \gamma_k^{n_k})
\]

The fear is that we have a situation where 

\begin{itemize}
    \item $\alpha \gamma_1 = \delta_1$
    \item $\alpha \gamma_2 = \delta_2$
    \item $\beta \gamma_1 = \delta_2$
    \item $\beta \gamma_2 = \delta_1$
\end{itemize}

so that ``accidentally'' $\alpha(\gamma_1 \gamma_2) = \beta(\gamma_1 \gamma_2)$.

But these kinds of situations are not possible because $\rho$ implies that
$\alpha$ and $\beta$ are (strongly) injective!

Indeed, recall that $\alpha$ and $\beta$ give injections from $\{ \gamma_1 \ldots, \gamma_k \}$
to $\{ \delta_1, \ldots, \delta_k \}$, which are in fact bijections since we're
dealing with finite sets of the same cardinality. 

Moreover, by assumption $\rho$ provides an inverse for $\alpha$ \emph{and} 
for $\beta$!

Then $\alpha$ and $\beta$ must be the same map on this set, and in particular
each $\gamma_i$ satisfies $\alpha \gamma_i = \beta \gamma_i$, as desired.

\section{Conclusion}

So, where did we start, and where did we end?

Well, by Beck's (Co)monadicity Theorem, we now know that $A$ is an
equivalence of categories from $\mathsf{Gph}$ to the category of coalgebras
in $\mathsf{Grp}$. In particular, this tells us that some group homomorphism
$f$ is of the form $A \varphi$ for some graph homomorphism $\varphi$
exactly when $f$ is additionally a coalgebra cohomomorphism!

Of course, we know how to check if something is a cohomomorphism. It's exactly
the condition outlined at the start of this paper.

% \newpage
% \bibliographystyle{plain}
% \bibliography{\string~/.bib.bib}

\end{document}
